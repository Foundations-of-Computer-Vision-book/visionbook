\chapter{Representing the input}

\reviewcomment{This chapter is currently just a sketch of an idea... it might end up being a section within another chapter or it might develop on a full chapter.}


Briefly, let's consolidate the multiple ways in which we can represent the input to a vision system.


%\section{Images}

An image can be represented in different ways making explicit certain aspects of the information present on the input. Here we will discuss on how can we describe the image itself, with a minimum processing. How can we represented the array of pixel intensities recorded by the camera.

{\bf Ordered array}. The simplest and most direct representation for an image is an ordered array of pixel intensities or colors:
\begin{equation}
\mathbf{s} = 
\left[
\begin{matrix}
s [1,1] & \dots & \\
\vdots & s [n,m] & \vdots \\
 & \dots  & s [N,M] \\
\end{matrix}
\right]
%\left[x [n,m] \right]
\end{equation}
Each value is a sample on a regular spatial grid. In this notation $s$ represents the pixel intensity at one location. This is the representation we are most used to and the typical used when taking a signal processing perspective.

{\bf Unordered set of points}. Another representation is a collection of points indicating its color and location explicitly:
\begin{equation}
S = \left\{ [s_i,x_i,y_i] : i \right\}
\end{equation}
this representation is commonly used when we want to make geometry explicit. $s_i$ is the pixel intensity (or color) recorded at location $(x_i,y_i)$. With this representation, we can apply geometric transformations easily by directly working with the spatial coordinates. Although both previous representations might seem equivalent, the set representation allows easily dealing with other image geometries where the points are not on a regular array. 

{\bf A function}. The image in represented as a continuous function whose input is
a location $(x, y)$ and it output is an intensity or a color, $s$:
\begin{equation}
   s = f_{\theta}(x,y)
\end{equation}
this representation is commonly used when we want to make image priors more explicit. The function $f_{\theta}$ is parameterized by the coefficients $\theta$. This function become specially interesting when the parameters $\theta$ are different than the original pixel values. They will have to be estimated from the original image. But once learned, the function $f$ should be able to take as input continuous spatial variables.

These three representations induce different ways of thinking about architectures to process the visual input. 

These representations are not opposed and can be used simultaneously.

The ordered array of pixels is the format that computers take as input when visualizing images. Therefore, it is always important to be familiar on how to transform any representation into an ordered array.


\section{Computing similarities}


%\section{Depth cameras}

%\section{Light-field}

%The input can also be a light field, or the whole light wave.



