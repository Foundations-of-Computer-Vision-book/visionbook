\chapter{Clustering and perceptual organization}

We humans see the world as organized into different levels of perceptual structure: contours and surfaces, objects and events. 

Many perceptual structures can be thought of as \textit{groups of primitive elements}: a contour is a group of points that form a line, an object is a group of parts that form a cohesive, nameable whole, etc. The study of \textbf{perceptual organization} therefore largely revolves around the study of \textbf{perceptual grouping}.

This leads us to a central question: which primitives should be grouped? We will examine a few answers below, including that primitives should be grouped based on \textit{similarity} or based on \textit{association}. We will also look at how natural groups can emerge as a byproduct of a downstream purpose.

Before we start, what, mathematically, do we mean by ``perceptual group"? Generally this refers to a neuron or some other indicator variable that selectively responds only to particular conjunctions of primitives.

\subsection{Grouping by similarity}


\subsection{Grouping by association}


\subsection{Emergent groups}